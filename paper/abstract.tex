	\begin{abstract}
		%
	\end{abstract}



%Equation \eqref{eq:second_order} is typically obtained by assembling the use the Finite Element Method matrices.
%where $M$ is called the mass matrix, (...).
%used to analyze engineering problems that require solving systems of differential equations. By use of set propagation techniques, uncertainties in the initial conditions or input functions can be solved 

%The Finite Element Method (FEM) is the gold standard for spatial discretization in numerical simulations for a wide spectrum of real-world engineering problems.
%
%Prototypical areas of interest include linear heat transfer and linear structural dynamics problems modeled with linear partial differential equations (PDEs).
%
%While different algorithms for direct integration of the equations of motion exist, exploring all feasible behaviors for varying loads, initial states and fluxes in models with large numbers of degrees of freedom remains a challenging task.
%
%In this article we propose a novel approach, based in set propagation methods and motivated by recent advances in the field of Reachability Analysis.
%
%Assuming a set of initial states and input states, the proposed method consists in the construction of a union of sets (flowpipe) that enclose the infinite number of solutions of the spatially discretized PDE.
%
%We present the numerical results obtained in four examples to illustrate the capabilities of the approach, and draw some comparisons with respect to reference numerical integration methods.
%
%We conclude that the proposed method presents specific and promising advantages, but the full potential of reachability analysis in solid mechanics problems is yet to be explored.